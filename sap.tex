\documentclass[
  bibliography=totoc]{scrreprt}

\usepackage{lmodern}


\usepackage{amssymb,amsmath}
\usepackage{ifxetex,ifluatex}
\usepackage{fixltx2e} % provides \textsubscript
\usepackage{vhistory}
\usepackage[affil-it]{authblk}

\ifnum 0\ifxetex 1\fi\ifluatex 1\fi=0 % if pdftex
  \usepackage[T1]{fontenc}
  \usepackage[utf8]{inputenc}


\else % if luatex or xelatex
  \ifxetex
    \usepackage{mathspec}
  \else
    \usepackage{fontspec}
  \fi
  \defaultfontfeatures{Ligatures=TeX,Scale=MatchLowercase}
\fi
% use upquote if available, for straight quotes in verbatim environments
\IfFileExists{upquote.sty}{\usepackage{upquote}}{}
% use microtype if available
\IfFileExists{microtype.sty}{%
\usepackage{microtype}
\UseMicrotypeSet[protrusion]{basicmath} % disable protrusion for tt fonts
}{}
\usepackage{hyperref}
\hypersetup{unicode=true,
            pdftitle={Statistical Analysis Plan},
            pdfauthor={James Totterdell; Julie Marsh},
            pdfborder={0 0 0},
            breaklinks=true}
\urlstyle{same}  % don't use monospace font for urls
\usepackage{natbib}
\bibliographystyle{plainnat}
\usepackage{longtable,booktabs}

\usepackage{vhistory}

\usepackage{graphicx}

\usepackage{grffile}
\makeatletter
\def\maxwidth{\ifdim\Gin@nat@width>\linewidth\linewidth\else\Gin@nat@width\fi}
\def\maxheight{\ifdim\Gin@nat@height>\textheight\textheight\else\Gin@nat@height\fi}
\makeatother
% Scale images if necessary, so that they will not overflow the page
% margins by default, and it is still possible to overwrite the defaults
% using explicit options in \includegraphics[width, height, ...]{}
\setkeys{Gin}{width=\maxwidth,height=\maxheight,keepaspectratio}
\IfFileExists{parskip.sty}{%
\usepackage{parskip}
}{% else
\setlength{\parindent}{0pt}
\setlength{\parskip}{6pt plus 2pt minus 1pt}
}
\setlength{\emergencystretch}{3em}  % prevent overfull lines
\providecommand{\tightlist}{%
  \setlength{\itemsep}{0pt}\setlength{\parskip}{0pt}}
\setcounter{secnumdepth}{5}
% Redefines (sub)paragraphs to behave more like sections
\ifx\paragraph\undefined\else
\let\oldparagraph\paragraph
\renewcommand{\paragraph}[1]{\oldparagraph{#1}\mbox{}}
\fi
\ifx\subparagraph\undefined\else
\let\oldsubparagraph\subparagraph
\renewcommand{\subparagraph}[1]{\oldsubparagraph{#1}\mbox{}}
\fi

%%% Use protect on footnotes to avoid problems with footnotes in titles
\let\rmarkdownfootnote\footnote%
\def\footnote{\protect\rmarkdownfootnote}


  \title{Statistical Analysis Plan}
  \subtitle{AuTOMATIC: Adaptive Trial of MessAging to improve Immunisation Coverage}
  \author{James Totterdell\textsuperscript{1} \\ Julie Marsh\textsuperscript{2}}
  
  \affil{\textsuperscript{1}University of Sydney, Health and Clinical Analytics, \textsuperscript{2}Telethon Kids Institute, Adaptive Health Intelligence}

    \date{Date: 20 October 2020}


\usepackage{scrlayer}
\DeclareNewLayer[% define a new layer
  foreground,
  foot,
  contents={\parbox[b][\layerheight][b]{\layerwidth}{%
    \hfill\includegraphics{assets/images/tki_ahi_logo}%
  }}
]{titlepage.logos}




\usepackage{float} \usepackage{setspace} \floatplacement{figure}{H}
\usepackage{flafter}
\usepackage{booktabs}
\usepackage{longtable}
\usepackage{array}
\usepackage{multirow}
\usepackage{wrapfig}
\usepackage{float}
\usepackage{colortbl}
\usepackage{pdflscape}
\usepackage{tabu}
\usepackage{threeparttable}
\usepackage{threeparttablex}
\usepackage[normalem]{ulem}
\usepackage{makecell}

\newcommand{\titledate}[2][2.5in]{%
  \noindent%
  \begin{tabular}{@{}p{#1}@{}}
    \\ \hline \\[-.75\normalbaselineskip]
    #2
  \end{tabular} \hspace{1in}
  \begin{tabular}{@{}p{#1}@{}}
    \\ \hline \\[-.75\normalbaselineskip]
    Date
  \end{tabular}\vspace{4\baselineskip}
}

\begin{document}
\AddLayersToPageStyle{empty}{titlepage.logos}% add the layer to pagestyle empty

\begin{titlepage}

  \makebox[0pt][l]{\smash{%
    \parbox[t][\dimexpr\textheight-\ht\strutbox\relax][t]{\textwidth}{%
      \includegraphics[width=3cm]{assets/images/waves-green-lighter}\hfill%
  }}}

  \begin{center}
    {\sectfont\Huge Statistical Analysis Plan\par}\vspace{3\baselineskip}
    {\huge AuTOMATIC: Adaptive Trial of MessAging to improve Immunisation Coverage\par}\vspace{3\baselineskip}
  \end{center}
  {\large Date: 20 October 2020\par}\vspace{0.5\baselineskip}
  {\large Version: 1.0\par}\vspace{0.5\baselineskip}
  {\large ANZCTR Number: U1111-1189-6054\par}\vspace{0.5\baselineskip}
  {\large           James Totterdell\textsuperscript{1},
          Julie Marsh\textsuperscript{2}  \par}\vspace{0.5\baselineskip}
  {\large\          \textit{\textsuperscript{1}University of Sydney, Health and Clinical Analytics, \textsuperscript{2}Telethon Kids Institute, Adaptive Health Intelligence}  \par}\vspace{8\baselineskip}

  \titledate{Statistician}
  \titledate{CPI}\vfill

\end{titlepage}

\RemoveLayersFromPageStyle{empty}{titlepage.logos}% remove the layer



\begin{versionhistory}
      \vhEntry{1.0}{2020-10-16}{JT}{Final version, protocol 2.0}
  \end{versionhistory}

{
    \setcounter{tocdepth}{1}
  \tableofcontents
}



\setstretch{1.5}

\hypertarget{definitions}{%
\chapter{Definitions}\label{definitions}}

\begin{table}[H]
\centering\begingroup\fontsize{10}{12}\selectfont

\begin{tabular}{l>{\raggedright\arraybackslash}p{25em}}
\toprule
Term & Definition\\
\midrule
Estimand & Estimands align protocol objectives with the quantification of the intervention effect(s). \newline     It is defined as “the target of estimation to address the scientific question of interest posed by the trial objective” in ICH E9(R1).\newline \newline     An estimand is a combination of eligibility criteria (population of interest), endpoint definition, treatment description (including the comparator), statistical analysis, treatment of intercurrent events post-randomisation (eg. missing values, non-compliance, use of rescue medication or interventions, etc.) and a population level summary (e.g. pairwise differences in the log hazard rates between the interventions).\newline \\
Index child & The first child scheduled for vaccination for a given parent after the parent has been randomised.\newline \\
Index vaccine & The first scheduled vaccine date for the index child after the parent’s randomisation date.\newline \\
Late vaccination & A child who has not received the recommended vaccinations within 4 weeks (28 days) of the scheduled date according to the standard childhood immunisation schedule.\newline \\
Parent & A parent of a child including a person who is regarded as the legally responsible caregiver of the child under law.\newline \\
Sites & Participating SmartVax GPs and community-based providers.\newline \\
Subsequent child & Any child belonging to a parent whom, at the moment of enrolment, already has another child enrolled.\newline \\
Under-vaccination & A child who has not received all of the age-appropriate vaccinations recommended by the standard childhood immunisation schedule.\newline \\
\bottomrule
\end{tabular}
\endgroup{}
\end{table}

\hypertarget{introduction}{%
\chapter{Introduction}\label{introduction}}

The purpose of this document is to describe the statistical methods which will be used to analyse data in the AuTOMATIC: Adaptive Trial of MessAging to improve Immunisation Coverage trial.
It is written for statisticians and should be read in conjunction with the protocol.

The basic trial structure, objectives, endpoints, and quantities of interest are outlined in Sections \ref{study-details} and \ref{study-design}, the eligibility criteria are explained in \ref{trial-population}, the statistical model and quantities of interest are introduced in Section \ref{analysis-methods}, trial adaptations are defined in Section \ref{interim-analyses-and-trial-adaptations}, and a summary of trial operating characteristics or provided in Section \ref{operating-characteristics}.

This statistical analysis plan (SAP) is based on version 2.0 of the study protocol, October 2020.

\hypertarget{study-details}{%
\chapter{Study Details}\label{study-details}}

\hypertarget{background-and-rationale}{%
\section{Background and Rationale}\label{background-and-rationale}}

The potential of SMS and the effect of message framing and timing on vaccination timeliness has not been studied in an Australian setting.
We hypothesise that SMS reminders emanating from a family's usual vaccine provider may be more effective for improving vaccination uptake and timeliness than, for example, impersonal messages originating from a government source.

The study aims to determine the optimal timing of sending reminders, as sending an SMS reminder to everyone before the scheduled-date may have a similar effectiveness to targeting parents of children who are over-due.
However, from a practical perspective, reducing the costs associated with the sending of SMS reminders will have an impact on how they are implemented in routine clinical care.
In addition, the optimal framing of the messages will be examined, as a neutral SMS reminder may be all that is necessary to convince parents to vaccinate on time.
All messages were assessed by a panel of consumer representatives prior to obtaining ethics approval.

\hypertarget{intervention}{%
\section{Intervention}\label{intervention}}

SmartVax is a vaccine safety initiative originally developed to monitor adverse reactions following vaccination.
The system integrates with all major GP patient information systems and sends an automated SMS to parents 3 days after receiving a vaccine to ask whether their child had any reactions (e.g.~fever, rash).
Parents can respond to the SMS and are prompted to complete an electronic survey asking for details of the side-effects if they indicated yes (``Y'').

Recently, SmartVax has been further developed to send optional automated SMS `pre-call' reminders to notify parents that their child's next vaccination is nearly due, due or over-due.
This study will investigate the effectiveness of sending SMS reminders using this technology.
The SmartVax platform will interrogate the medical records of children registered at a practice to determine when vaccinations are due, and to send an automated SMS reminder prompting parents to call the clinic and schedule an appointment at the appropriate time.
Following vaccine administration, the SmartVax platform will be able to use the details of the vaccination (entered electronically by site staff) and the child's date of birth to schedule the next SMS reminder.
The platform will be used to send SMS reminders at different time points and with different message framing.
We will examine the effect of these interventions on vaccine uptake.
The intervention is designed to influence the parent, therefore, each parent will be allocated at random to a single intervention arm, which will apply to any children under their care at all scheduled vaccine occasions.

Twelve intervention arms, consisting of the combination of four different message framings and three different timings will be investigated (i.e.~a 4x3 factorial design).
The text message framings are classified as either:

\begin{enumerate}
\def\labelenumi{\arabic{enumi}.}
\tightlist
\item
  positive in tone (espousing a personal benefit from vaccination),
\item
  negative in tone (espousing a risk from late or failed vaccination),
\item
  neutral in tone (factual and conveying neither benefit nor risk), or
\item
  social norm (conveying the societal preponderance of vaccination).
\end{enumerate}

The messages may be sent at timings:

\begin{enumerate}
\def\labelenumi{\arabic{enumi}.}
\tightlist
\item
  14 days before the scheduled due date,
\item
  on the scheduled due date,
\item
  7 days after the schedule due date.
\end{enumerate}

Additionally, a control arm that receives no SmartVax SMS reminder will be included as a reference arm.
This control arm will still receive standard care from the practice.
Actual standard care may vary slightly from practice to practice and may include, for example, a letter from Medicare to signal an overdue vaccination, a notice from Centrelink if the parent is receiving a childcare rebate or a telephone/written reminder from the GP clinic.
Parents in the standard care group will not receive an unsolicited SMS reminder for an upcoming vaccination, but participants may receive an appointment reminder or confirmation via SMS if they have scheduled an appointment if that is usual practice for their provider.

\hypertarget{study-objectives}{%
\section{Study Objectives}\label{study-objectives}}

The aim of the study is to determine whether provider-initiated SMS reminders are effective for improving the timeliness of routine vaccination among Australian children.
The parent (whose behaviour we seek to change) can only be considered naive to the intervention on the first child-vaccination occasion, therefore, the primary analysis will be performed on the outcome of the first intervention occasion for each parent.

The specific objectives and the related outcomes are tabled below.

\hypertarget{primary-objectives}{%
\subsection{Primary Objectives}\label{primary-objectives}}

\begingroup\fontsize{10}{12}\selectfont

\begin{tabu} to \linewidth {>{\raggedright}X>{\raggedright}X}
\toprule
Objective & Outcome\\
\midrule
To determine the effect of different timing and framing of a personalised SMS reminder, emanating from a family’s vaccine provider (general practice or community vaccination clinic), on the proportion of children vaccinated within 28 days of the scheduled due date for routine childhood vaccination compared to usual practice.
The objective of most interest is identification of the best message framing and timing combination out of those considered in terms of proportion of children vaccinated within 28 days of the scheduled due date. & Vaccination status (vaccinated or unvaccinated) at 28 days after scheduled vaccine due date for the index child for the index vaccine. On occasion, vaccination might occur prior to the due date. Such occasions will be counted as vaccinated by 28 days after the due date if it occurs no more than 14 days before the scheduled due date.\\
 & Parents may have multiple children eligible for study inclusion and each child may receive multiple vaccine doses during the trial period. The index child is the first child scheduled to be vaccinated after the parent’s randomisation date and the index vaccine is the first scheduled vaccine date for each child after the parent’s randomisation date\\
\bottomrule
\end{tabu}
\endgroup{}

\hypertarget{secondary-objectives}{%
\subsection{Secondary Objectives}\label{secondary-objectives}}

\begingroup\fontsize{10}{12}\selectfont

\begin{tabu} to \linewidth {>{\raggedright}X>{\raggedright}X}
\toprule
Objective & Outcome\\
\midrule
To determine the effectiveness of vaccine provider initiated SMS reminders of varying content and timing  for reducing the time to vaccination, relative to vaccine due date, for routine childhood vaccines & Time to vaccination (measured in days) from 14 (timing of earliest intervention) days before the scheduled due date up to 42 days after the scheduled due date. Calculated as the difference between date of vaccine administration and due date as recorded in the SmartVax system for the \textit{index vaccination} of each parent randomised. Events occuring more than 42 days after the scheduled due date are considred right censored\\
\addlinespace\addlinespace
 & Time to vaccination (measured in days) from 14 days before the scheduled due date (timing of earliest intervention) up to 42 days after the scheduled due date. Calculated as the difference between date of vaccine administration and due date as recorded in the SmartVax system for \textit{all scheduled vaccinations} of each parent randomised. Events occuring more than 42 days after the scheduled due date are right censored\\
\addlinespace\addlinespace
To further evaluate the effectiveness of vaccine provider initiated SMS reminders of varying content and timing  for improving the rate of timely vaccination for routine childhood vaccines & Vaccination status (vaccinated or not) by 28 days after the scheduled due date measured by the difference between date of vaccine administration and due date as recorded in the SmartVax system for \textit{all scheduled vaccinations} of each parent randomised\\
\bottomrule
\end{tabu}
\endgroup{}

\hypertarget{study-design}{%
\chapter{Study Design}\label{study-design}}

\hypertarget{type}{%
\section{Type}\label{type}}

This is a Bayesian adaptive, factorial, superiority trial.
Frequent interim analyses will be performed to assess if a combination of message framing and timing is superior to the other interventions.
Intervention performance with respect to the primary outcome will be used to inform response adaptive randomisation leading to a higher allocation of future participants to better performing interventions.
Interventions which perform worse than control will be dropped from the trial.
If any intervention is better than control, or on average receiving an intervention is better than receiving control, the control group will be dropped.

In addition to the response adaptive randomisation, pre-defined decision rules for early stopping are:

\begin{enumerate}
\def\labelenumi{\arabic{enumi})}
\tightlist
\item
  Stop for superiority if there is substantial evidence that one intervention arm results in a higher 28 day vaccination proportion than all others including control.
\item
  Stop for harm if there is substantial evidence that all intervention arms have a lower 28 day vaccination proportion than control or there is strong evidence that on average an intervention is worse than control.
\end{enumerate}

\hypertarget{estimands}{%
\section{Estimands}\label{estimands}}

Estimands provide a structured framework to increase the transparency and precision in describing an intervention effect of interest.
They inform decision making within a clinical trial setting by clearly describing the risks and benefits of an intervention.
Interventions designed to increase childhood vaccine uptake and coverage are assessed in terms of timely percentage uptake and time-to-event endpoints, respectively.
However, these endpoints address different objectives and the effect of each intervention may differ over time or by the age of the child at scheduled vaccination.
It is anticipated that the results may differ between the estimands as each addresses a different hypothesis.
Estimands and sensitivity analyses are described according to recommendations in ICH E9-R1.

\hypertarget{primary-estimand}{%
\subsection{Primary Estimand}\label{primary-estimand}}

\textbf{Objective:} To determine the real-world comparative effectiveness of the first vaccine-provider initiated SMS reminder framing and timing for increasing the proportion of children vaccinated within 28 days of scheduled routine vaccine date.

\textbf{Strategy:} Treatment-policy, irrespective of age-at-scheduled-vaccine.

\textbf{Population of interest:} Parents of index children receiving their index vaccine, in addition to meeting the study eligibility criteria in Section \ref{eligibility-criteria}.
No details are provided from participating sites concerning any reasons for delayed child vaccine administration or vaccines administered outside the site.

\textbf{Endpoint:} Evidence of administration of vaccine for index vaccine at participating site within 28 days of scheduled vaccine date (binary endpoint).

\textbf{Treatment description:} Control arm and 12 intervention arms as detailed in Section \ref{intervention}.
Comparisons will be made between all arms.

\textbf{Treatment of intercurrent events:} Parents will be included in the intervention group they were allocated to, irrespective of whether they received the SMS text or materials related to the site standard-of-care or not.
In the absence of evidence of vaccine administration within 28 days of scheduled date at the participating site, including due to participants who are lost to follow up or move out of the GP-site catchment area, it will be assumed that the vaccine has not been administered.

\textbf{Statistical method:} Bayesian logistic regression; details in Section \ref{analysis-of-primary-outcome}.

\textbf{Population summary:} The posterior probability that an intervention is superior to all other invention arms.
In addition, posterior summaries from the statistical model parameters for average SMS text framing and timing, relative to the overall intervention effect and interaction effects for framing and timing combinations, will be presented as point estimates and highest density intervals (HDI).
Further details are given in Section \ref{analysis-of-primary-outcome}.

\textbf{Sensitivity analysis:} The sensitivity of the results will be investigated by varying the model covariates.
For example, the interaction terms will be excluded, and main effects terms excluded for comparison of results.
Further details are given in Section \ref{sensitivity-analyses-of-primary-outcome}.

\hypertarget{secondary-estimand}{%
\subsection{Secondary Estimand}\label{secondary-estimand}}

\textbf{Objective:} To determine the real-world timeliness of first vaccine-provider initiated SMS reminder framing and timing for reducing the time to vaccination for routine childhood vaccines.

\textbf{Strategy:} Treatment-policy, irrespective of age-at-scheduled-vaccine.

\textbf{Population of interest:} Parents of index children receiving their index vaccine, in addition to meeting the study eligibility criteria in Section \ref{eligibility-criteria}.
No details are provided from participating sites concerning any reasons for delayed child vaccine administration or vaccines administered outside the site.

\textbf{Endpoint:} Time to vaccination at participating sites from the date of randomisation.
Vaccine administration which occurs later than 42 days after the vaccine due date will be censored.
The reminder system records no information as to which of the scheduled vaccines have been given.
Therefore, vaccinations which occur within 14 days of a subsequently scheduled vaccination are assumed to correspond to the upcoming scheduled dose rather than the preceding one.

\textbf{Treatment description:} Control arm and 12 intervention arms as detailed in Section \ref{intervention}.

\textbf{Treatment of intercurrent events:} Parents will be included in the intervention group they were allocated to, irrespective of whether they received the SMS text or materials related to the site standard-of-care or not.
In the absence of evidence of vaccine administration within 42 days of scheduled date at the participating sites, including due to participants who are lost to follow up or move out of the GP-site catchment area, it will be assumed that the vaccine has not been administered and the endpoint will be right censored at 42 days after the due date.

\textbf{Statistical method:} Bayesian proportional continuation ratio regression model; more details in Section \ref{analysis-of-secondary-outcomes}.

\textbf{Population summary:} The posterior summary of conditional odds ratio and restricted mean survival time under each intervention relative to the control group.

\hypertarget{randomisation}{%
\section{Randomisation}\label{randomisation}}

The parent is the unit of randomisation, as it is their behaviour we seek to change.
Parents of eligible children will therefore be allocated at random to one of the 13 arms.
Each child may receive more than one scheduled vaccination during the study period, but the parent will receive the same intervention for all eligible children under their care for all scheduled vaccinations.

To avoid randomising the same parent to different arms across different sites (in case they have registered at multiple SmartVax GPs and/or community-based providers) the random allocation will be automatically generated from a central de-identified allocation list.
The use of centralised randomisation will ensure that if the parent is registered at multiple participating SmartVax sites, they will be allocated to the same intervention arm for all eligible children under their care, at all scheduled vaccine doses and across all participating sites throughout the study period.
The standard de-identification algorithm employed across all sites will ensure that the same unique parent ID will be generated at all sites based on the recorded mobile phone number.
It is anticipated that there will be a low rate of incorrect or out of date mobile phone numbers recorded in the site records and that this will be equally likely to occur across all intervention arms.

Prior to the first interim analysis, parents will be equally likely to be randomised to any of the interventions and a higher ratio will be allocated to the control arm; this is the run-in period before response-adaptive randomisation commences.
Following each interim analysis, the allocation ratios to each intervention arm will be updated and a new allocation list generated on the middleware application.
Any arms which have been dropped will receive zero allocations and the control arm allocation ratio will be fixed unless dropped.
All parents randomised after an interim will then be allocated to arms according to the new allocation list.

\hypertarget{sample-size}{%
\section{Sample size}\label{sample-size}}

The minimum sample size will be 1,500 index vaccinations due to the timing of the first interim analysis.
A maximum of 10,000 parents will be randomised unless a pre-specified stopping rule is met at interim analyses.
This sample size was chosen based on simulations of the adaptive trial (see Section \ref{operating-characteristics}).

\hypertarget{trial-population}{%
\chapter{Trial Population}\label{trial-population}}

Up to 10,000 parents will be enrolled from GP and public vaccination clinics participating in SmartVax across Australia.
A diverse geographical range of clinics will be invited to participate to ensure representation from a broad socio-demographic cross section of suburbs, including from regional Australia.
Agreements will be sought from all clinics, and governing bodies prior to enrolling at these sites.
Study participation (parent and child/children) is from parent's randomisation until the last scheduled vaccine dose has been administered before five years of age for any child under their care or until the trial concludes.

\hypertarget{eligibility-criteria}{%
\section{Eligibility criteria}\label{eligibility-criteria}}

To be included in the study ALL of the following criteria must be satisfied;

\begin{itemize}
\tightlist
\item
  Parents of children aged 6 weeks -- 4 years (strictly less than 5 years) who are registered with a SmartVax registered GP clinic or community vaccination clinic or whose clinic has expressed interest in SmartVax.
\item
  Parents must have a mobile phone number registered with the vaccine provider.
\item
  Eligible children must have their details entered into their electronic health record, including the parent mobile phone number and the child's date of birth and name.
\end{itemize}

The participant will be excluded if ANY of the following apply:

\begin{itemize}
\tightlist
\item
  The parent(s) of the child have previously requested not to be contacted by the clinic via SMS.
\item
  Parents who in the opinion of clinic staff would be unsuitable for inclusion in the study, for example because they are known to attend for routine vaccinations elsewhere, they have relocated outside of the clinic catchment area, the registered mobile phone number is known to be obsolete or wrong, or because they are registered as conscientious objectors to vaccination.
\item
  The critical information required to produce a unique identification number has not been entered properly into the practice's electronic medical record (i.e.~parent mobile phone number, child's date of birth, child's first and surname).
\item
  Children known or suspected to be twins and triplets will be excluded; producing a unique identification number will not be possible for siblings with the same birthdate.
\end{itemize}

It may be necessary to withdraw the following participants from the analysis if:

\begin{itemize}
\tightlist
\item
  Any of the exclusion criteria are met subsequent to enrolment and before the end of follow-up.
\end{itemize}

\hypertarget{analysis-datasets}{%
\section{Analysis datasets}\label{analysis-datasets}}

The primary analysis dataset consists of index vaccines for each parent.
The secondary outcome datasets will consist of all children and all vaccines for each parent.
At the time of an interim analysis, all index vaccinations which are past due (greater than 28 days after the scheduled due date) relative to the analysis reference date will be included in the analysis set.
The reference date will be one week prior to the date on which the analysis is being run to allow the system to catch-up on any outstanding updates.

The data will be analysed and reported on an intention-to-treat (ITT) basis with all randomised participants contributing according to the estimands defined in \ref{estimands}.
In particular, we cannot know that the SMS reminders were received or read by the participants, only that they were delivered.
Participants will be analysed in the arm they were allocated to.

\hypertarget{analysis-methods}{%
\chapter{Analysis Methods}\label{analysis-methods}}

\hypertarget{data}{%
\section{Data}\label{data}}

Individual-level de-identified and encrypted data for every enrolled parent with associated child vaccination outcomes will be exported electronically and in an encrypted fashion to the secure REDCap database on the Telethon Kids Institute server.
This data will include:

\begin{enumerate}
\def\labelenumi{\arabic{enumi}.}
\tightlist
\item
  the child's date of birth;
\item
  practice code;
\item
  child and parent's unique identifier codes;
\item
  date and time of randomisation;
\item
  intervention allocation (message type and timing);
\item
  date and time of intervention SMS sent if sent;
\item
  date and time of any subsequent index vaccination at that clinic within 90 days after randomisation.
\item
  Parent postcode
\item
  Date vaccine due
\item
  SMS delivery failure (if applicable)
\end{enumerate}

For each scheduled vaccination, the closest vaccination occasion in time will be obtained from SmartVax and the relative timing of the vaccination will be calculated as the difference between the date of vaccine administration and the scheduled due date.

Vaccinations occurring later than 42 days after the scheduled due date will be right-censored.
Vaccinations occurring earlier than 14 days before the scheduled due date will be excluded.
Therefore, the time to vaccination in days will range from -14 to 42 days.
Vaccination status at 28 days will be calculated as the proportion of participants with a time to vaccination between -14 to 28 days relative to the scheduled due date.

\hypertarget{analysis-of-primary-outcome}{%
\section{Analysis of primary outcome}\label{analysis-of-primary-outcome}}

For each arm and overall, the number of assigned participants and the raw count and proportion amongst those participants vaccinated with 28 days will be reported.

Inferences for the primary outcome will be based on a Bayesian logistic regression model.
The model will be used to estimate the log-odds of vaccination by 28 days (primary endpoint) following the scheduled due date for the interventions.
The model will account for variation in outcomes by:

\begin{itemize}
\tightlist
\item
  intervention effect (timing by framing combination)
\item
  scheduled age of vaccination at 2,4,6,12,18, and 48 months of age (fixed categorical effects).
\item
  clinic attended (random effect)
\item
  calendar time of vaccine due date (grouped into epochs of 4 weeks and smoothed across groups).
\end{itemize}

The factorial design of the intervention arms in terms of SMS message framing and timing is presented in Table \ref{tab:intsum} with cell labels for each framing-timing combination.

\begin{table}[H]

\caption{\label{tab:intsum}Intervention summary}
\centering
\fontsize{10}{12}\selectfont
\begin{tabular}[t]{lrrrl}
\toprule
  & Arm & Timing & Framing & Combination\\
\midrule
Control & 0 & 0 & 0 & $a_0b_0$\\
Intervention 1 & 1 & 1 & 1 & $a_1$$b_1$\\
Intervention 2 & 2 & 1 & 2 & $a_1$$b_2$\\
Intervention 3 & 3 & 1 & 3 & $a_1$$b_3$\\
Intervention 4 & 4 & 1 & 4 & $a_1$$b_4$\\
Intervention 5 & 5 & 2 & 1 & $a_2$$b_1$\\
Intervention 6 & 6 & 2 & 2 & $a_2$$b_2$\\
Intervention 7 & 7 & 2 & 3 & $a_2$$b_3$\\
Intervention 8 & 8 & 2 & 4 & $a_2$$b_4$\\
Intervention 9 & 9 & 3 & 1 & $a_3$$b_1$\\
Intervention 10 & 10 & 3 & 2 & $a_3$$b_2$\\
Intervention 11 & 11 & 3 & 3 & $a_3$$b_3$\\
Intervention 12 & 12 & 3 & 4 & $a_3$$b_4$\\
\bottomrule
\end{tabular}
\end{table}

In what follows, the message timings are denoted by \(a\) and message framings by \(b\).
The interim analyses are designated by \(t=1,...,T\) and cohort \(t\) refers to individuals recruited between interim \(t-1\) and \(t\) which has sample size \(n_t\).
The total sample size at analysis \(t\) is then \(N_t = \sum_{j=1}^t n_t\).
We denote all data available at analysis \(t\) by \(D_t\).

The model for the log-odds of vaccination at 28 days after the scheduled due date, denoted \(\eta\), and the associated probability, \(p\), are modelled by
\[
\begin{aligned}
\eta_i &= \alpha + l_i\zeta + x_{ia}^\mathsf{T}\gamma_a + x_{ib}^\mathsf{T}\gamma_b + x_{iab}^\mathsf{T}\gamma_{ab} + w_i^\mathsf{T}\beta + z_{i\xi}^\mathsf{T}\xi + z_{i\tau}^{\mathsf{T}}\tau\\
p_i &= \text{logit}^{-1}(\eta_i).
\end{aligned}
\]

\begin{itemize}
\tightlist
\item
  The \(\alpha\) term is the intercept parameter giving the average log-odds of vaccination for the control arm.
\item
  The \(\zeta\) term is the average intervention effect relative to the control arm across all message framings and timings.
\item
  The \(\gamma_a\) constrained to sum to zero, are the deviation effects for the message timings from the average intervention effect.
\item
  The \(\gamma_b\) constrained to sum to zero, are the deviation effects for the message framings from the average intervention effect.
\item
  The \(\gamma_{ab}\) constrained to sum to zero on the relevant message and timing margin, are the interaction effects between message framing and timing.
\item
  The \(\beta\) are other model parameters for the model covariates: scheduled age of vaccination.
\item
  The \(\xi\) and \(\tau\) are random effect terms adjusting for clinic and calendar due date respectively.
\end{itemize}

The prior distributions for the intercept and average intervention effect parameters will be set to
\[
\begin{aligned}
\alpha &\sim N(\text{logit}^{-1}(0.8), 2.5) \\
\zeta &\sim N(0, 1)
\end{aligned}
\]
noting that we expect the proportion vaccinated by 28 days to be closer to 1 than 0.

The priors on the \(\gamma\) terms will be set so as to enforce the relevant sum-to-zero constraints
\[
\begin{aligned}
S_l &= \mathrm{I}_l - l^{-1}\mathrm{J}_l \\
\gamma_a &\sim N\left(0, S_4\right) \\
\gamma_b &\sim N\left(0, S_3\right) \\
\gamma_{ab} &\sim N\left(0, S_4\otimes S_3\right).
\end{aligned}
\]
The prior on other model coefficients will be
\[
\beta \sim N(0, 2.5^2).
\]

The use of response adaptive randomisation means that allocation ratios will change over time.
If background trends in the outcome are also present, then arms with higher allocations over certain periods may display larger or smaller effects than those attributable to the intervention.
Therefore, the two random effects terms are included, one for clinic, and one for epoch.
These will be modelled as
\[
\begin{aligned}
\xi|\sigma_xi^2 &\sim N(0, \sigma^2_\xi) \\
\sigma_\xi^2 &\sim \text{Half-}t(3, 0, 1) \\
\tau_e|\sigma_\tau^2 &\sim N(\tau_{e-1}, \sigma_\tau^2) \\
\sigma_\tau^2 &\sim \text{Half-}t(3, 0, 1) \\
\tau_1 &= 0.
\end{aligned}
\]
where \(e\) indicates the epoch.
The parameter \(\tau\) has a first-order random walk prior to smooth baseline changes in the response across epoch's.
The epoch groupings will be in 28 day increments with \(\tau_1=1\) representing the most recent 28 days.
In the event that accrual is slower then expected, this grouping may need to be reviewed.

Pre-specified adaptations will be assessed using the primary analysis model.

\hypertarget{sensitivity-analyses-of-primary-outcome}{%
\section{Sensitivity analyses of primary outcome}\label{sensitivity-analyses-of-primary-outcome}}

As sensitivity analyses, reduced versions of the primary analysis model will be estimated.
A reduced model is specified without adjustment for clinic and epoch.
Another reduced model is specified without the interaction term between framing and timing, and also without the framing and timing effects themselves for a completely pooled model comparing control to intervention.

The reduced models considered for comparison will be
\[
\begin{aligned}
\eta &= 1\alpha + L\zeta + W\beta + Z_\xi\xi + Z_\tau\tau \\
\eta &= 1\alpha + L\zeta + X_a\gamma_a + X_b\gamma_b + W\beta + Z_\xi\xi + Z_\tau\tau \\
\eta &= 1\alpha + L\zeta + X_a\gamma_a + X_b\gamma_b + X_{ab}\gamma_{ab} + W\beta.
\end{aligned}
\]
The same priors on common parameters as for the primary analysis will be used in these reduced models.

\hypertarget{statistical-quantities}{%
\section{Statistical quantities}\label{statistical-quantities}}

The posterior distribution of the model parameters will be approximated using Markov chain Monte Carlo methods via the Stan language.
At least 10,000 draws will be generated from the joint posterior distribution for the primary analysis, and these draws will be used to approximate the quantities of interest as outlined in the current section and Section \ref{interim-analyses-and-trial-adaptations}.

For each of the 13 arms, we define \(\mu_j\) as the conditional log-odds of vaccination by 28 days for arm \(j=0,...,12\) and define \(\theta_{jk} = \mu_j - \mu_k\) as the pairwise differences for \(j,k=0,...,12\).

\hypertarget{posterior-summaries}{%
\subsection{Posterior summaries}\label{posterior-summaries}}

At the final analysis, posterior summaries for the model parameters will be presented in terms of point estimates and highest density intervals.
These values will summarise: the average intervention effect across all message framing and timings relative to control, the average effect of each framing and timing relative to the overall intervention effect and control, the main effect of each framing and timing, and interaction effects of the framing and timing combinations.
Due to the potential to drop the control arm, comparisons will also be made with the current most probably best arm as defined in the next section.
Where explicit hypotheses and decision rules have been stated, these will be assessed and reported in terms of their posterior probability as outlined below.

\hypertarget{probability-an-intervention-arm-is-best}{%
\subsection{Probability an intervention arm is best}\label{probability-an-intervention-arm-is-best}}

Decisions related to the primary outcome and response adaptive randomisation will be based on the posterior probability that each arm is superior to all others.
We define \(\pi_{jt}\) be the posterior probability an intervention \(j=1,...,12\), is the best intervention given the data available up to interim \(t\).
The set of interventions to be included in the comparison is only those interventions which are still active (those not found to be worse than control).
This set of interventions is denoted \(\mathcal{A}_t\).
If \(\mathcal{A}_t=\varnothing\), that is, there are no active interventions, then the trial will be stopped.
The value is
\[
\begin{aligned}
\pi_{jt} &= \mathbb P[\mu_j > \mu_k;\forall k\ne j,k,j\in\mathcal{A}_t|D_t].
\end{aligned}
\]
The value of \(\pi_{jt}\) for any \(j\notin\mathcal{A}_t\) is defined to be 0.
The \textbf{current most probably best} intervention arm is then chosen to be intervention arm which satisfies \(b_t = \arg\max_j \pi_{jt}\).

\hypertarget{probability-an-intervention-arm-is-better-than-control}{%
\subsection{Probability an intervention arm is better than control}\label{probability-an-intervention-arm-is-better-than-control}}

To declare an intervention arm superior, it must also be better than standard of care.
Let \(\phi_{jt}\) be the posterior probability that each intervention \(j=1,...,12\), is better than control given the data available to interim \(t\), given by
\[
\begin{aligned}
\phi_{jt} &= \mathbb P[\mu_j > \mu_0|D_t].
\end{aligned}
\]

\hypertarget{probability-of-beneficial-average-intervention-effect}{%
\subsection{Probability of beneficial average intervention effect}\label{probability-of-beneficial-average-intervention-effect}}

There may be insufficient precision to declare any single intervention better than control, but there may be strong evidence that on average receiving an intervention is better than not.
The probability that an intervention is on average beneficial is
\[
\begin{aligned}
\varphi_t &= \mathbb P[\alpha_1 > 0 | D_t].
\end{aligned}
\]

\hypertarget{probability-of-rank}{%
\subsection{Probability of Rank}\label{probability-of-rank}}

At the end of the trial, we may not have declared any intervention arm superior.
There may still be a subset of intervention arms which are competing amongst each other, but standout as superior to the others.
Investigation of rank probabilities may help with decision making in such a scenario.

At each interim, the cell means and factor level means may be ranked according to (in terms of cell means as an example)
\[
\begin{aligned}
R_j &= \text{rank}(\mu_j) = \sum_{l=1}^{12} \mathbb I[\mu_j\geq\mu_l] \\
R_j^\star &=\text{rank}\left(\mathbb E[R_j|D_t]\right).
\end{aligned}
\]
For each intervention arm and main effect, we will investigate the marginal rank probabilities \(\mathbb P[R_j=k|D_t]\), that is, the probability that arm \(j\) is ranked \(k\)th in response rate.

Clusters of ranked arms may be identified by collections which have high probability of exceeding a given rank and near-zero probability of being below this rank.
For example, four arms may each have probability 1 of rank at least 4 and 0 of rank less than 4 indicating that these stand-out as the top 4 interventions even though there is insufficient information to declare any one of those 4 best overall.

\hypertarget{analysis-of-secondary-outcomes}{%
\section{Analysis of secondary outcomes}\label{analysis-of-secondary-outcomes}}

\hypertarget{secondary-outcome-1.---time-to-event}{%
\subsection{Secondary Outcome 1. - Time-to-event}\label{secondary-outcome-1.---time-to-event}}

For the secondary outcome definitions refer to Section \ref{study-objectives}.

For each arm and overall, the empirical hazard and survival for days to vaccination will be reported.

Inferences for the time to vaccination outcome will be based on a proportional continuation ratio model (logistic link).
Primary interest lies in the treatment effect on the conditional log-odds of vaccination at a given time (or the corresponding conditional odds ratio).

Events may occur at times \(k\in\{1,...,K\}\) with \(y_{ik}=1\) denoting vaccination and \(y_{ik}=0\) no vaccination.
We take \(k=1\) to be 14 days before the due date of the index vaccination.
SMS messages may then be sent at times \(k=1, 15, 22\) (14 days before due, on the due date, and 7 days after the due date, respectively).
Censoring at 42 days after the due date then occurs at \(k=57\).

The interventions themselves are time-varying; the message cannot affect the outcome until it has been sent.
All participants contribute to the baseline hazard until the their intervention becomes active.
The intervention design matrix are the same as for the primary outcome.
If \(\tau_i\in\{1,15,22\}\) denotes the timing of the intervention for participant \(i\), then
\[
u_{ik} = \mathbf{1}_{[\tau_i, \infty)}(k)
\]
indicates whether the intervention is active at time \(k\) for participant \(i\).

We specify a semi-parametric hazard regression model for flexible modelling of the baseline (control group) hazard with the curve modelled by penalised O'Sullivan splines \citep{eilers1996, currie2002, wand2008}
\[
\begin{aligned}
\eta_{ik} &= s_{0k} + u_{ik}\left[l_i\zeta +  x_{ia}^\mathsf{T}\gamma_a + x_{ib}^\mathsf{T}\gamma_b + x_{iab}^\mathsf{T}\gamma_{ab} \right] + w_i^\top\beta + z^\mathsf{T}_{\xi,i}\xi + z^\mathsf{T}_{\tau,i}\tau \\
\lambda_{ik} &= \text{logit}^{-1}(\eta_{ik}) \\
s_{0k} &= \alpha_{0} + \alpha_{1} k + \textstyle\sum_{m=1}^M b_{m} B_{mk} \\
b_m|\sigma^2 &\sim N(0,\sigma^2) \\
\sigma &\sim \text{Half-}t(3, 0, 10) \\
\alpha &\sim N(0,2.5^2).
\end{aligned}
\]

where \(B_m\) are spline bases enforcing the penalty and priors on the other parameters are as in the primary analysis model.
We specify 28 equally spaced interior knots to be used for the baseline hazard.

\hypertarget{sensitivity-analyses}{%
\subsubsection{Sensitivity Analyses}\label{sensitivity-analyses}}

It may be that the assumption of proportional continuation ratio is inappropriate.
As sensitivity analysis, expanded versions of the model will be investigated.
These models will allow interventions effects to be flexibly varying on the conditional log-odds of vaccination.
Under these models, the intervention specific curves are specified as
\[
\eta_{ik} = s_{0k} + u_{ik}\left[s_{j(i),k}\right] + w_i^\top\beta + z^\mathsf{T}_{\xi,i}\xi + z^\mathsf{T}_{\tau,i}\tau
\]
where the group specific term is modelled by either
\[
\begin{aligned}
(1)\quad s_{jk} &= \gamma_{0j} + \gamma_{1j}k \\
(2)\quad s_{jk} &= \gamma_{0j(a)} + \gamma_{1j(a)}k + \textstyle\sum_{m=1}^M g_{j(a)m} B_{mk} \\
\gamma &\sim N(0,2.5^2) \\
g_{jm}|\omega_j^2 &\sim N(0, \omega_j^2) \\
\omega_j &\sim \text{Half-}t(3, 0, 10).
\end{aligned}
\]
where \(j(a)\) returns the timing of intervention \(j\).
That is \((1)\) represents combination specific curves and \((2)\) timing specific curves combined across all message framings.

Under these models, the proportional continuation ratio does not apply.
Therefore, primary interest will be in the survival curves themselves and restricted mean survival time (RMST) \citep{royston2011, royston2013} to \(t^\star\) in each group, defined as
\[
\text{RMST}_j(k^\star) = \int_0^{k^\star} S_j(k)\ \mathrm dk,\quad j=0,1,....,12.
\]
with \(S_j(k)\) the survival curve of remaining unvaccinated obtained from the above models for group \(j\).
In particular, we will compare groups according to the difference in their RMST relative to the baseline (no intervention) survival
\[
\Delta_j(k^\star) = \int_0^{k^\star} S_j(k) - S_0(k)\ \mathrm dk,\quad j=1,...,12.
\]
as a summary of the relative average number of days spent unvaccinated.

\hypertarget{secondary-outcomes-2.-and-3.}{%
\subsection{Secondary Outcomes 2. and 3.}\label{secondary-outcomes-2.-and-3.}}

Secondary outcome 2. and 3. will be analysed analogously to the primary and first secondary outcome.
However, nesting of vaccine occasions within parents will be accounted for by an additional hierarchical component for the parent in the linear predictor.

\hypertarget{interim-analyses-and-trial-adaptations}{%
\section{Interim analyses and trial adaptations}\label{interim-analyses-and-trial-adaptations}}

A first interim analysis is scheduled to occur when 1,500 index vaccinations reach the primary endpoint.
Subsequent interim analyses are scheduled to occur every additional 500 index vaccinations which reach the primary endpoint.
Only the primary analysis will be performed at each interim and thus only the primary outcome will inform the response-adaptive randomisation and trial adaptations.
At each interim analysis there will be index cases who are enrolled but yet to reach 28 days after their due vaccination; these are not eligible for inclusion in the current interim analysis but will be included in subsequent interims once their 28 days has passed.
With the maximum sample size of 10,000, a first interim analysis at 1,500, and interims every 500 participants implies a maximum of 18 interim analyses.

\hypertarget{response-adaptive-randomisation}{%
\subsection{Response Adaptive Randomisation}\label{response-adaptive-randomisation}}

The control arm will have fixed allocation of \(q_0=1/5\) throughout the duration of the trial unless dropped in which case \(q_0=0\).
This fixed allocation applies regardless of the number of active intervention arms.
The higher allocation to control was chosen to increase power on the multiple comparisons of each intervention to control for effectiveness.

At the start of the trial, allocation ratios to each intervention arm will be equal to \(\frac{4}{5}\times\frac{1}{12}\).
The randomisation probabilities to the intervention arms will be updated is data accrues.

Following each analysis \(t\), the allocation probability to arm \(j\) will be a function of the probability that each arm is the best intervention arm in proportion to
\[
r_{jt} \propto \sqrt{\frac{\pi_{jt}\mathbb V[\mu_j|D_t]}{n_{jt} + 1}},\quad j=1,...,12.
\]
The ratios are normalised to sum to one and share the remaining probability after accounting for the fixed control arm allocation \(q_0\)
\[
q_{jt} = (1 - q_0)\frac{r_{jt}}{\sum_j r_{jt}},\quad j=1,...,12.
\]

An intervention arm may be permanently dropped if there is evidence it is worse than no intervention.
If \(\hat\phi_{jt}<\kappa_t^{\text{harmful}}\) then \(q_{jt}=0\) and it's allocation mass is redistributed amongst the remaining intervention arms.

\hypertarget{intervention-effectiveness-and-superiority}{%
\subsection{Intervention effectiveness and superiority}\label{intervention-effectiveness-and-superiority}}

At any analysis, if a single arm has at least \(\kappa_t^{\text{effective}}\) posterior probability of being better than control then the control arm will be dropped.
Additionally, if on average receiving an intervention is more effective than control, then the control arm will be dropped.
The dropping rule for the control arm is,
\[
\left(\exists j\in\{1,...,12\}:\hat\phi_{jt}>\kappa_t^{\text{effective}}\right)\text{ or } \left(\hat\varphi_t>\kappa_t^{\text{effective}}\right).
\]

At any analysis, if a single arm has at least \(\kappa_t^{\text{superior}}\) posterior probability of being the best overall arm, and this arm is better than the control group with at least \(\kappa_t^{\text{effective}}\) posterior probability, then this result triggers a stopping of the trial for superiority of that arm.
The stopping rule is,
\[
\exists \ j\in\{1,...,12\} : \hat\pi_{jt}>\kappa_t^{\text{superior}} \text{ and }\ \hat\phi_{jt}>\kappa_t^{\text{effective}}.
\]

\hypertarget{harmful-interventions}{%
\subsection{Harmful interventions}\label{harmful-interventions}}

If at any interim analysis an intervention is found to be harmful, then that intervention will be dropped.
The dropping rule for intervention arm \(j\) is
\[
\hat\phi_{jt}<\kappa_t^{\text{harmful}},\ j=1,...,12.
\]

If all arms are worse than control then the trial may be stopped early for lack of effectiveness of any intervention relative to standard care.
Alternatively, if on average receiving an intervention is worse than control the trial may be stopped for lack of effectiveness.
The decision rule is
\[
\left(\hat\phi_{jt}<\kappa_t^{\text{harmful}}\text{ for all }j\in\{1,...,12\}\right)\text{ or }\ \left(\hat\varphi_t<\kappa_t^{\text{harmful}}\right).
\]

\hypertarget{thresholds}{%
\subsection{Thresholds}\label{thresholds}}

Due to the number of arms and the number of interims, the thresholds are set to relatively strict values to account for multiplicity.
The thresholds are also allowed to reduce as information accrues to balance the risk of making a decision based on less information.
The threshold to be used in the interim analyses are
\[
\begin{aligned}
\kappa_t^{\text{harmfjul}} &= 0.01{\sqrt{\frac{N_t}{{10,000}}}} \\
\kappa_t^{\text{superior}} &= 0.95 - 0.25\sqrt{\frac{N_t}{10,000}} \\
\kappa_t^{\text{effective}} &= 0.99^{\sqrt{\frac{N_t}{{10,000}}}}
\end{aligned},\quad t = 1,...,17,
\]
where \(N_t\) is the sample size at the interim analysis.

\hypertarget{summary-of-interim-analyses}{%
\subsection{Summary of Interim Analyses}\label{summary-of-interim-analyses}}

The following summarises the adaptation process at each interim analysis.

\begin{itemize}
\tightlist
\item
  The primary analysis model is updated conditional on the available data as outlined in Section \ref{analysis-of-primary-outcome}.
\item
  The posterior summaries and statistical quantities are calculated as outlined in Section \ref{statistical-quantities}.
\item
  The decision rules are evaluated as outlined in Section \ref{interim-analyses-and-trial-adaptations} and Table \ref{tab:decrules}.
\item
  If the trial has not stopped, the allocation probabilities are updated as in Section \ref{response-adaptive-randomisation}.
\end{itemize}

\begin{table}[H]

\caption{\label{tab:decrules}Interim analysis, $t$, decision rule summary}
\centering
\fontsize{10}{12}\selectfont
\begin{tabular}[t]{llll}
\toprule
Comparison & Criteria & Decision & Action\\
\midrule
j vs. control & $\phi_{jt}>\kappa_t^{\text{effective}}$ & Intervention effective & Drop control arm\\
avg vs. control & $\varphi_t > \kappa_t^{\text{effective}}$ & On average effective & Drop control arm\\
j vs. control & $\phi_{jt}<\kappa_t^{\text{harmful}}$ & Intervention harmful & Drop intervention $j$\\
avg vs. control & $\varphi_t < \kappa_t^{\text{harmful}}$ & On average harmful & Stop trial\\
all vs. control & $\mathcal{A}_t=\varnothing$ & No active interventions & Stop trial\\
j vs. all & $\pi_{jt}>\kappa_t^{\text{superior}} \cap \phi_{jt}>\kappa_t^{\text{effective}}$ & Intervention superior & Stop trial\\
\bottomrule
\end{tabular}
\end{table}

\hypertarget{subgroup-analyses}{%
\section{Subgroup analyses}\label{subgroup-analyses}}

The effect of SMS vaccine reminders on the primary endpoint will be investigated in the final analysis for the following subgroups:

\begin{itemize}
\tightlist
\item
  Scheduled age for vaccination (2, 4, 6, 12, 18 months and 48 months)
\end{itemize}

\hypertarget{missing-data}{%
\section{Missing data}\label{missing-data}}

Randomisation will only occur for participants with complete baseline data (all data fields except for the date of index vaccination).
The SV software will include basic logic checks to prohibit entry of nonsensical or internally inconsistent data.
Children with missing date of index vaccination data after 42 days since randomisation will be assumed to have not received the index vaccine before that date.
No attempt will be made to confirm the practice-entered data with the practice or to ascertain or verify from other sources.

\hypertarget{software}{%
\section{Software}\label{software}}

Data processing will be performed using R.
Models will be fit in R using Stan via the \texttt{rstan} package.

\hypertarget{operating-characteristics}{%
\chapter{Operating Characteristics}\label{operating-characteristics}}

Given the trial adaptations, simulations were required to explore and assess the operating characteristics of the design.
Extensive simulations were undertaken where the trial design parameters were varied and trial quantities of interest were investigated.

The choice of the trial parameters and thresholds aims to achieve the following:

\begin{itemize}
\tightlist
\item
  \(\approx 0.25\) probability of dropping the control arm in the null scenario, \(\mu_0 = \mu_1 = ... = \mu_{12}\)
\item
  \(\approx 0.05\) probability of declaring any single intervention superior at stopping in the null scenario
\item
  \(\approx 0.10\) probability of declaring any single intervention effective at stopping in the null scenario
\end{itemize}

Later dropping/stopping times were preferred to earlier times, therefore the thresholds were scaled with the information ratio.

For efficiency, simulations utilised variational approximations to the model parameter posteriors, and posterior quantities were calculated on the basis of this approximation.
Operating characteristics were determined by conducting 10,000 simulations under each scenario.
The trial parameters as outlined in this statistical analysis plan were chosen based on the results from these simulations.

The following does not include all simulations conducted, but rather only those under the chosen parameterisation.

\hypertarget{simulation-scenarios}{%
\section{Simulation scenarios}\label{simulation-scenarios}}

Although additional simulation scenarios were considered, included in this document are results under the following scenarios.
Effect sizes of odds ratios range from \(1/1.2\) to \(1.5\) by increments of 0.1.
Under each effect size we consider:

\begin{itemize}
\tightlist
\item
  all interventions are modified by effect size,
\item
  one intervention is modified by effect size,
\item
  one message timing is modified by effect size,
\item
  one message framing is modified by effect size.
\end{itemize}

The simulations are summarised with a focus on the timing of the decision rules being triggered and the posterior summaries at trial stopping.

\clearpage

\hypertarget{simulation-results}{%
\section{Simulation results}\label{simulation-results}}

\hypertarget{all-equally-effective}{%
\subsection{All equally effective}\label{all-equally-effective}}

\hypertarget{average-effect-decisions}{%
\subsubsection{Average Effect Decisions}\label{average-effect-decisions}}

\begin{figure}
\centering
\includegraphics{sap_files/figure-latex/alleffdecavg-1.pdf}
\caption{\label{fig:alleffdecavg}Decision times for average intervention effect.}
\end{figure}

\clearpage

\hypertarget{intervention-effect-decisions}{%
\subsubsection{Intervention Effect Decisions}\label{intervention-effect-decisions}}

\begin{figure}
\centering
\includegraphics{sap_files/figure-latex/unnamed-chunk-4-1.pdf}
\caption{\label{fig:unnamed-chunk-4}Decision times for superiority of each intervention.}
\end{figure}

\begin{figure}
\centering
\includegraphics{sap_files/figure-latex/unnamed-chunk-5-1.pdf}
\caption{\label{fig:unnamed-chunk-5}Decision times for effectiveness of each intervention.}
\end{figure}

\begin{figure}
\centering
\includegraphics{sap_files/figure-latex/unnamed-chunk-6-1.pdf}
\caption{\label{fig:unnamed-chunk-6}Decision times for harmfulness of each intervention.}
\end{figure}

\begin{figure}
\centering
\includegraphics{sap_files/figure-latex/unnamed-chunk-7-1.pdf}
\caption{\label{fig:unnamed-chunk-7}Earliest decision times across all intervention arms.}
\end{figure}

\clearpage

\hypertarget{trial-stopping}{%
\subsubsection{Trial Stopping}\label{trial-stopping}}

\begin{figure}
\centering
\includegraphics{sap_files/figure-latex/alleffdecbest-1.pdf}
\caption{\label{fig:alleffdecbest}Stopping times and reason for stopping.}
\end{figure}

\clearpage

\hypertarget{arm-dropping}{%
\subsubsection{Arm Dropping}\label{arm-dropping}}

\begin{figure}
\centering
\includegraphics{sap_files/figure-latex/alleffdecdrp-1.pdf}
\caption{\label{fig:alleffdecdrp}Decision times for dropping each intervention arm.}
\end{figure}

\clearpage

\hypertarget{arm-summaries}{%
\subsubsection{Arm Summaries}\label{arm-summaries}}

\begin{figure}
\centering
\includegraphics{sap_files/figure-latex/unnamed-chunk-9-1.pdf}
\caption{\label{fig:unnamed-chunk-9}Expected sample size by arm.}
\end{figure}

\begin{figure}
\centering
\includegraphics{sap_files/figure-latex/unnamed-chunk-10-1.pdf}
\caption{\label{fig:unnamed-chunk-10}Expected posterior mean by arm.}
\end{figure}

\begin{figure}
\centering
\includegraphics{sap_files/figure-latex/unnamed-chunk-11-1.pdf}
\caption{\label{fig:unnamed-chunk-11}Expected posterior probability of superiority by arm.}
\end{figure}

\begin{figure}
\centering
\includegraphics{sap_files/figure-latex/unnamed-chunk-12-1.pdf}
\caption{\label{fig:unnamed-chunk-12}Expected posterior probability of effectiveness by arm.}
\end{figure}

\clearpage

\hypertarget{contrast-summaries}{%
\subsubsection{Contrast Summaries}\label{contrast-summaries}}

\begin{figure}
\centering
\includegraphics{sap_files/figure-latex/unnamed-chunk-14-1.pdf}
\caption{\label{fig:unnamed-chunk-14}Expected posterior summary of contrasts.}
\end{figure}

\begin{figure}
\centering
\includegraphics{sap_files/figure-latex/unnamed-chunk-15-1.pdf}
\caption{\label{fig:unnamed-chunk-15}Expected posterior probability of contrast effect.}
\end{figure}

\clearpage

\hypertarget{one-intervention-effective}{%
\subsection{One intervention effective}\label{one-intervention-effective}}

\hypertarget{average-effect-decisions-1}{%
\subsubsection{Average Effect Decisions}\label{average-effect-decisions-1}}

\begin{figure}
\centering
\includegraphics{sap_files/figure-latex/oneeffdecavg-1.pdf}
\caption{\label{fig:oneeffdecavg}Decision times for average intervention effect.}
\end{figure}

\hypertarget{intervention-effect-decisions-1}{%
\subsubsection{Intervention Effect Decisions}\label{intervention-effect-decisions-1}}

\begin{figure}
\centering
\includegraphics{sap_files/figure-latex/unnamed-chunk-16-1.pdf}
\caption{\label{fig:unnamed-chunk-16}Decision times for superiority of each intervention.}
\end{figure}

\begin{figure}
\centering
\includegraphics{sap_files/figure-latex/unnamed-chunk-17-1.pdf}
\caption{\label{fig:unnamed-chunk-17}Decision times for effectiveness of each intervention.}
\end{figure}

\begin{figure}
\centering
\includegraphics{sap_files/figure-latex/unnamed-chunk-18-1.pdf}
\caption{\label{fig:unnamed-chunk-18}Decision times for harmfulness of each intervention.}
\end{figure}

\begin{figure}
\centering
\includegraphics{sap_files/figure-latex/unnamed-chunk-19-1.pdf}
\caption{\label{fig:unnamed-chunk-19}Earliest decision times across all intervention arms.}
\end{figure}

\clearpage

\hypertarget{trial-stopping-1}{%
\subsubsection{Trial Stopping}\label{trial-stopping-1}}

\begin{figure}
\centering
\includegraphics{sap_files/figure-latex/unnamed-chunk-20-1.pdf}
\caption{\label{fig:unnamed-chunk-20}Stopping times and reason for stopping.}
\end{figure}

\clearpage

\hypertarget{arm-dropping-1}{%
\subsubsection{Arm Dropping}\label{arm-dropping-1}}

\begin{figure}
\centering
\includegraphics{sap_files/figure-latex/unnamed-chunk-21-1.pdf}
\caption{\label{fig:unnamed-chunk-21}Decision times for dropping each intervention arm.}
\end{figure}

\clearpage

\hypertarget{arm-summaries-1}{%
\subsubsection{Arm Summaries}\label{arm-summaries-1}}

\begin{figure}
\centering
\includegraphics{sap_files/figure-latex/unnamed-chunk-22-1.pdf}
\caption{\label{fig:unnamed-chunk-22}Expected sample size by arm.}
\end{figure}

\begin{figure}
\centering
\includegraphics{sap_files/figure-latex/unnamed-chunk-23-1.pdf}
\caption{\label{fig:unnamed-chunk-23}Expected posterior mean by arm.}
\end{figure}

\begin{figure}
\centering
\includegraphics{sap_files/figure-latex/unnamed-chunk-24-1.pdf}
\caption{\label{fig:unnamed-chunk-24}Expected posterior probability of superiority by arm.}
\end{figure}

\begin{figure}
\centering
\includegraphics{sap_files/figure-latex/unnamed-chunk-25-1.pdf}
\caption{\label{fig:unnamed-chunk-25}Expected posterior probability of effectiveness by arm.}
\end{figure}

\clearpage

\hypertarget{contrast-summaries-1}{%
\subsubsection{Contrast Summaries}\label{contrast-summaries-1}}

\begin{figure}
\centering
\includegraphics{sap_files/figure-latex/unnamed-chunk-26-1.pdf}
\caption{\label{fig:unnamed-chunk-26}Expected posterior summary of contrasts.}
\end{figure}

\begin{figure}
\centering
\includegraphics{sap_files/figure-latex/unnamed-chunk-27-1.pdf}
\caption{\label{fig:unnamed-chunk-27}Expected posterior probability of contrast effect.}
\end{figure}

\clearpage

\hypertarget{one-timing-effective}{%
\subsection{One timing effective}\label{one-timing-effective}}

\hypertarget{average-effect-decisions-2}{%
\subsubsection{Average Effect Decisions}\label{average-effect-decisions-2}}

\begin{figure}
\centering
\includegraphics{sap_files/figure-latex/timeffdecavg-1.pdf}
\caption{\label{fig:timeffdecavg}Decision times for average intervention effect.}
\end{figure}

\clearpage

\hypertarget{intervention-effect-decisions-2}{%
\subsubsection{Intervention Effect Decisions}\label{intervention-effect-decisions-2}}

\begin{figure}
\centering
\includegraphics{sap_files/figure-latex/unnamed-chunk-28-1.pdf}
\caption{\label{fig:unnamed-chunk-28}Decision times for superiority of each intervention.}
\end{figure}

\begin{figure}
\centering
\includegraphics{sap_files/figure-latex/unnamed-chunk-29-1.pdf}
\caption{\label{fig:unnamed-chunk-29}Decision times for effectiveness of each intervention.}
\end{figure}

\begin{figure}
\centering
\includegraphics{sap_files/figure-latex/unnamed-chunk-30-1.pdf}
\caption{\label{fig:unnamed-chunk-30}Decision times for harmfulness of each intervention.}
\end{figure}

\begin{figure}
\centering
\includegraphics{sap_files/figure-latex/unnamed-chunk-31-1.pdf}
\caption{\label{fig:unnamed-chunk-31}Earliest decision times across all intervention arms.}
\end{figure}

\clearpage

\hypertarget{trial-stopping-2}{%
\subsubsection{Trial Stopping}\label{trial-stopping-2}}

\begin{figure}
\centering
\includegraphics{sap_files/figure-latex/unnamed-chunk-32-1.pdf}
\caption{\label{fig:unnamed-chunk-32}Stopping times and reason for stopping.}
\end{figure}

\clearpage

\hypertarget{arm-dropping-2}{%
\subsubsection{Arm Dropping}\label{arm-dropping-2}}

\begin{figure}
\centering
\includegraphics{sap_files/figure-latex/unnamed-chunk-33-1.pdf}
\caption{\label{fig:unnamed-chunk-33}Decision times for dropping each intervention arm.}
\end{figure}

\clearpage

\hypertarget{arm-summaries-2}{%
\subsubsection{Arm Summaries}\label{arm-summaries-2}}

\begin{figure}
\centering
\includegraphics{sap_files/figure-latex/unnamed-chunk-34-1.pdf}
\caption{\label{fig:unnamed-chunk-34}Expected sample size by arm.}
\end{figure}

\begin{figure}
\centering
\includegraphics{sap_files/figure-latex/unnamed-chunk-35-1.pdf}
\caption{\label{fig:unnamed-chunk-35}Expected posterior mean by arm.}
\end{figure}

\begin{figure}
\centering
\includegraphics{sap_files/figure-latex/unnamed-chunk-36-1.pdf}
\caption{\label{fig:unnamed-chunk-36}Expected posterior probability of superiority by arm.}
\end{figure}

\begin{figure}
\centering
\includegraphics{sap_files/figure-latex/unnamed-chunk-37-1.pdf}
\caption{\label{fig:unnamed-chunk-37}Expected posterior probability of effectiveness by arm.}
\end{figure}

\clearpage

\hypertarget{contrast-summaries-2}{%
\subsubsection{Contrast Summaries}\label{contrast-summaries-2}}

\begin{figure}
\centering
\includegraphics{sap_files/figure-latex/unnamed-chunk-38-1.pdf}
\caption{\label{fig:unnamed-chunk-38}Expected posterior summary of contrasts.}
\end{figure}

\begin{figure}
\centering
\includegraphics{sap_files/figure-latex/unnamed-chunk-39-1.pdf}
\caption{\label{fig:unnamed-chunk-39}Expected posterior probability of contrast effect.}
\end{figure}

\clearpage

\hypertarget{one-framing-effective}{%
\subsection{One framing effective}\label{one-framing-effective}}

\hypertarget{average-effect-decisions-3}{%
\subsubsection{Average Effect Decisions}\label{average-effect-decisions-3}}

\begin{figure}
\centering
\includegraphics{sap_files/figure-latex/meseffdecavg-1.pdf}
\caption{\label{fig:meseffdecavg}Decision times for average intervention effect.}
\end{figure}

\clearpage

\hypertarget{intervention-effect-decisions-3}{%
\subsubsection{Intervention Effect Decisions}\label{intervention-effect-decisions-3}}

\begin{figure}
\centering
\includegraphics{sap_files/figure-latex/unnamed-chunk-40-1.pdf}
\caption{\label{fig:unnamed-chunk-40}Decision times for superiority of each intervention.}
\end{figure}

\begin{figure}
\centering
\includegraphics{sap_files/figure-latex/unnamed-chunk-41-1.pdf}
\caption{\label{fig:unnamed-chunk-41}Decision times for effectiveness of each intervention.}
\end{figure}

\begin{figure}
\centering
\includegraphics{sap_files/figure-latex/unnamed-chunk-42-1.pdf}
\caption{\label{fig:unnamed-chunk-42}Decision times for harmfulness of each intervention.}
\end{figure}

\begin{figure}
\centering
\includegraphics{sap_files/figure-latex/unnamed-chunk-43-1.pdf}
\caption{\label{fig:unnamed-chunk-43}Earliest decision times across all intervention arms.}
\end{figure}

\clearpage

\hypertarget{trial-stopping-3}{%
\subsubsection{Trial Stopping}\label{trial-stopping-3}}

\begin{figure}
\centering
\includegraphics{sap_files/figure-latex/unnamed-chunk-44-1.pdf}
\caption{\label{fig:unnamed-chunk-44}Stopping times and reason for stopping.}
\end{figure}

\clearpage

\hypertarget{arm-dropping-3}{%
\subsubsection{Arm Dropping}\label{arm-dropping-3}}

\begin{figure}
\centering
\includegraphics{sap_files/figure-latex/unnamed-chunk-45-1.pdf}
\caption{\label{fig:unnamed-chunk-45}Decision times for dropping each intervention arm.}
\end{figure}

\clearpage

\hypertarget{arm-summaries-3}{%
\subsubsection{Arm Summaries}\label{arm-summaries-3}}

\begin{figure}
\centering
\includegraphics{sap_files/figure-latex/unnamed-chunk-46-1.pdf}
\caption{\label{fig:unnamed-chunk-46}Expected sample size by arm.}
\end{figure}

\begin{figure}
\centering
\includegraphics{sap_files/figure-latex/unnamed-chunk-47-1.pdf}
\caption{\label{fig:unnamed-chunk-47}Expected posterior mean by arm.}
\end{figure}

\begin{figure}
\centering
\includegraphics{sap_files/figure-latex/unnamed-chunk-48-1.pdf}
\caption{\label{fig:unnamed-chunk-48}Expected posterior probability of superiority by arm.}
\end{figure}

\begin{figure}
\centering
\includegraphics{sap_files/figure-latex/unnamed-chunk-49-1.pdf}
\caption{\label{fig:unnamed-chunk-49}Expected posterior probability of effectiveness by arm.}
\end{figure}

\clearpage

\hypertarget{contrast-summaries-3}{%
\subsubsection{Contrast Summaries}\label{contrast-summaries-3}}

\begin{figure}
\centering
\includegraphics{sap_files/figure-latex/unnamed-chunk-50-1.pdf}
\caption{\label{fig:unnamed-chunk-50}Expected posterior summary of contrasts.}
\end{figure}

\begin{figure}
\centering
\includegraphics{sap_files/figure-latex/unnamed-chunk-51-1.pdf}
\caption{\label{fig:unnamed-chunk-51}Expected posterior probability of contrast effect.}
\end{figure}

\renewcommand\refname{References}
\bibliography{assets/bib/sap.bib}



\end{document}
